\documentclass[a4paper,sffamily,12pt]{article}

\usepackage[T1]{fontenc}
\usepackage[french]{babel}
\usepackage[utf8]{inputenc}

\begin{document}

	\vspace{0.5cm}
	
%	\section{Dependances fonctionelles}
%	
%		\noindent- $adresse, ville \rightarrow franchise, nbsalle$ \\
%		- $adresse, ville, numSalle \rightarrow SallecompatibleEn3D, nbPlaceStandard,\\
%		 nbPlaceHandicape,nbDbox$ \\
%		- $nomFilm, dateSortie \rightarrow public, idPers AS idReal, duree, compatible3D$ \\
%		- $nomfilm, dateSortie, role \rightarrow idPers AS idAct$ \\
%		- $adresse, ville, horaire, dateProjection, numSalle \rightarrow nomFilm, dateSortie, diffusionen3D $ \\
%		- $nomC, prenomC, numReservation \rightarrow nbPlaceStandardRes, nbPlaceHandicapeRes,\\
%		 nbDboxRes, nomfilm, adresse, ville, salle, horaire$ \\
%		- $horaire, nomfilm, adresse, ville \rightarrow  numsalle$ \\ 
%		\\
%		- $idCine \rightarrow adresse, ville$ \\
%		- $idFilm \rightarrow nomFilm, dateSortie$ \\
%		- $idPers \rightarrow nomP, prenomP$ \\
%		- $idPers, idFilm \rightarrow role$ \\
%		- $idCine \rightarrow nbSalles$ \\
%		- $idClient \rightarrow nomC, prenomC$ \\
%		- $idSeance \rightarrow adresse, ville, horaire, dateSortie, numSalle$ \\
%		
%		\vspace{0.5cm}
%	
%	\section{Algo de Bernstein}
%	
%		\subsection{pas de 1}
%			
%			\vspace{0.5cm}
%			
%			\noindent1ere DF :\\
%			
%			\noindent{adresse}+={adresse} \\
%			{ville}+={ville} \\
%			--> nécéssaire \\
%			
%			\vspace{0.5cm}
%			
%			\noindent2eme DF : \\
%			
%			\noindent{adresse}+={adresse} \\
%			{ville}+={ville} \\
%			{numSalle}+={numSalle} \\
%			--> nécéssaire \\
%			
%			\vspace{0.5cm}
%			
%			\noindent3eme DF : \\
%			
%			\noindent{nomFilm}+={nomFilm} \\
%			{dateSortie}+={dateSortie} \\
%			--> nécéssaire \\
%			
%			\vspace{0.5cm}
%			
%			\noindent4eme DF : \\
%			
%			\noindent{nomFilm}+={nomFilm} \\
%			{dateSortie}+={dateSortie} \\
%			{role}+={role} \\
%			--> nécéssaire \\
%			
%			\vspace{0.5cm}
%			
%			\noindent5eme DF : \\
%			
%			\noindent{adresse}+={adresse} \\
%			{ville}+={ville} \\
%			{horaire}+={horaire} \\
%			{dateProjection}+={dateProjection} \\
%			{numSalle}+={numSalle} \\
%			--> nécéssaire \\
%			
%			\vspace{0.5cm}
%			
%			\noindent6eme DF : \\
%			
%			\noindent{nomClient}+={nomC} \\
%			{prenomClient}+={prenomC} \\
%			{numReservation}+={numReservation} \\
%			--> nécéssaire \\
%			
%			\vspace{0.5cm}
%			
%			\noindent7eme DF : \\
%			
%			\noindent{horaire}+={horaire} \\
%			{nomFilm}+={nomFilm} \\
%			--> nécéssaire
%			
%			\vspace{0.5cm}
%			
%			\noindent8eme DF : \\
%			
%			\noindent{horaire}+={horaire} \\
%			{idPers}+={idPers, nomP, prenomP} \\
%			{idFilm}+={idFilm, nomFilm, dateSortie, public, idPers AS idReal, duree, compatible3D} \\
%			
%			\vspace{0.5cm}
%		
%		\subsection{pas de 2}
%		
%			pas d'attributs superflus à gauche
%			
%			\vspace{0.5cm}
%		
%		\subsection{pas de 3}
%			
%			\vspace{0.5cm}
%			
%			\underline{adresse, ville} $+$ \\
%			1- franchise, nbSalles\\
%			
%			\vspace{0.5cm}
%			
%			\underline{adresse, ville, numSalle} $+$ \\
%			1- franchise, nbSalles, salleCompatible3D, nbPlaceHandicape, nbPlaceStandard, nbDbox\\
%			
%			\vspace{0.5cm}
%			
%			\underline{nomFilm, dateSortie} $+$ \\
%			1- public, idPers, duree, compatible3D\\
%			2- nomP, prenomP\\
%			
%			\vspace{0.5cm}
%			
%			\underline{nomFilm, dateSortie, role} $+$ \\
%			1- public, idPers, duree, compatible3D\\
%			2- nomP, prenomP\\
%			
%			\vspace{0.5cm}
%			
%			\underline{adresse, ville, horaire, datePojection, numSalle} $+$ \\
%			1- franchise, nbSalles, salleCompatible3D, nbPlaceHandicape, nbPlaceStandard, nbDbox, nomFilm, dateSortie, diffusionEn3D\\
%			2- public, idPers, duree, compatible3D\\
%			3- nomP, prenomP\\
%			
%			\vspace{0.5cm}
%			
%			\underline{nomC, prenomC, numReservation} $+$ \\
%			1- addresse, ville, horaire, dateProjection, numSalle, nbPlaceHandicape, nbPlaceStandard, nbDbox\\
%			2- franchise, nbSalle, nomFilm, dateSortie, difusionEn3D\\
%			3- public, idPers, duree, compatible3D\\
%			4- nomP, prenomP\\
%				
%			\vspace{0.5cm}
%			
%			\underline{horaire, nomFilm, adresse, ville} $+$ \\
%			1- franchise, numSalle, nbSalle\\
			
			\section{Dépendance fonctionnelle}
			
				\noindent- (1) idCine $\rightarrow$ adresse, ville \\
				- (2) adresse, ville $\rightarrow$ franchise, nbsalle \\
				- (3) idCine $\rightarrow$ franchise, nbSalles \\
				- (4) idCine, numSalle $\rightarrow$ SallecompatibleEn3D, nbPlaceStandard, nbPlaceHandicape,nbDbox \\
		 		- (5) idFilm $\rightarrow$ nomFilm, dateSortie \\
				- (6) nomFilm, dateSortie $\rightarrow$ public, idReal, duree, compatible3D \\
				- (7) idFilm, role $\rightarrow$  idAct \\
				- (8) idReal $\rightarrow$ nomR, prenomR \\
				- (9) idAct $\rightarrow$ nomA, prenomA \\
				- (10) idClient $\rightarrow$ nomC, prenomC \\
				- (11) idClient, numReservation $\rightarrow$ nbPlaceStandardRes, nbPlaceHandicapeRes, nbPlaceDBox, idSeance \\
				- (12) idSeance, idCine $\rightarrow$ horaire, dateProjection, numSalle, idFilm, diffusionEn3D \\
				
			\section{Algo de Bernstein}
	
				L'algo de Bernstein se fait en 4 parties : \\
					- Caclculer la CV(DF) et les clés. Si R est en 3FN, on s'arrête. \\
					- Partitionner CV(DF) e groupe DFi (1 <= i <= k) tels que toutes les df d'un même groupes aient la même partie gauche. \\
					- Construire un schéma <Ri(Ui), DFi> pour chaque groupe DFi, où Ui est l'ensemble des attribut apparaissant dans DFi. \\
					- Si aucun des schémas définis ne contient de clé X de R, rajouter un schéma <Rk+1(X), \{\}>. \\	
	
				\subsection{Caclcul de CV(DF)}
		
					La couverture minimal se fait en trois parties : \\
						- Toutes les dépendances doivent être élémentaire ; les décomposer si nécessaire. \\
						- Eliminer les attributs superflus du coté gauche de la df. \\
						- Eliminer les dfs redondantes.
					
					\subsubsection{pas 1}
		
						\noindent- (1) idCine $\rightarrow$ ville \\
						- (1) idCine $\rightarrow$ adresse \\
						- (2) adresse, ville $\rightarrow$ franchise \\
						- (2) adresse, ville $\rightarrow$ nbsalle \\
						- (3) idCine $\rightarrow$ franchise \\
						- (3) idCine $\rightarrow$ nbSalles \\
						- (4) idCine, numSalle $\rightarrow$ SallecompatibleEn3D \\
				 		- (4) idCine, numSalle $\rightarrow$ nbPlaceStandard \\
				 		- (4) idCine, numSalle $\rightarrow$ nbPlaceHandicape \\
				 		- (4) idCine, numSalle $\rightarrow$ nbDbox \\
				 		- (5) idFilm $\rightarrow$ nomFilm \\
				 		- (5) idFilm $\rightarrow$ dateSortie \\				 		
						- (6) nomFilm, dateSortie $\rightarrow$ public \\
						- (6) nomFilm, dateSortie $\rightarrow$ idReal \\
						- (6) nomFilm, dateSortie $\rightarrow$ duree \\
						- (6) nomFilm, dateSortie $\rightarrow$ compatible3D \\
						- (7)idFilm, role $\rightarrow idAct$ \\
						- (8) idReal $\rightarrow$ nomR \\
						- (8) idReal $\rightarrow$ prenomR \\						
						- (9) idAct $\rightarrow$ nomA \\
						- (9) idAct $\rightarrow$ prenomA \\						
						- (10) idClient $\rightarrow$ nomC \\
						- (10) idClient $\rightarrow$ prenomC \\						
						- (11) idClient, numReservation $\rightarrow$ idSeance \\
						- (11) idClient, numReservation $\rightarrow$ nbPlaceStandardRes \\
						- (11) idClient, numReservation $\rightarrow$ nbPlaceHandicapeRes \\
						- (11) idClient, numReservation $\rightarrow$ nbPlaceDBox \\
						- (12) idSeance, idCine $\rightarrow$ horaire \\
						- (12) idSeance, idCine $\rightarrow$ dateProjection \\
						- (12) idSeance, idCine $\rightarrow$ numSalle \\
						- (12) idSeance, idCine $\rightarrow$ idFilm \\
						- (12) idSeance  idCine $\rightarrow$ diffusionEn3D \\
										
					\subsubsection{pas2}
		
						\noindent - (2) adresse, ville $\rightarrow$ franchise, nbsalle \\
							\\
							\underline{adresse+} \\
							adresse \\
							\underline{ville+} \\
							ville \\
							\\
						$\rightarrow$ it's OK \\
		
						\noindent - (4) idCine, numSalle $\rightarrow$ SallecompatibleEn3D, nbPlaceStandard, nbPlaceHandicape,nbDbox \\
							\\
							\underline{idCine+} \\
							idCine \\
							adresse \\
							ville \\
							franchise \\
							nbSalle \\
							\underline{numSalle+} \\
							numSalle \\
							\\
						$\rightarrow$ it's OK \\
						
						\noindent - (6) nomFilm, dateSortie $\rightarrow$ public, idReal, duree, compatible3D \\																						\\
							\underline{nomFilm+} \\
							nomFilm \\
							\underline{dateSotie+} \\
							dateSortie \\
							\\
						$\rightarrow$ it's OK \\
					
						\noindent - (7) idFilm, role $\rightarrow$  idAct \\
							\\
							\underline{idFilm+} \\
							idFilm \\
							nomFilm \\
							dateSortie \\
							public \\
							idReal \\
							duree \\
							compatible3D \\
							nomA \\
							prenomA \\
							\underline{role+} \\
							role \\
							\\
						$\rightarrow$ it's OK \\
							
						\noindent - (11) idClient, numReservation $\rightarrow$ nbPlaceStandardRes, nbPlaceHandicapeRes, nbPlaceDBox, idSeance \\
							\\
							\underline{idClient+} \\
							idClient \\
							nomC \\
							prenomC \\
							\underline{numReservation+} \\
							numReservation \\	
							\\									
						$\rightarrow$ it's OK \\					
						
						\noindent - (12) idSeance, idCine $\rightarrow$ horaire, dateProjection, numSalle, idFilm, diffusionEn3D \\												
						\\
						\underline{idSeance+}
						idSeance \\
						\underline{idCine+}
						adresse \\
						ville \\
						franchise \\
						nbSalle \\
						\\
						$\rightarrow$ it's OK \\
		
					\subsubsection{pas3}		
			
						Eliminons tout d'abord les dfs qui sont préservées par transitivité : \\
			
							\noindent- $ 1 idCine \rightarrow adresse, ville$ \\
							- $2 adresse, ville \rightarrow franchise, nbsalle$ \\
							- $ 3 idCine \rightarrow franchise, nbSalles$ \\
							
							Si l'on prend les dfs 1, 2 et 3, on remarque que l'on peut supprimer la 3 car on peut retrouver celle-ci par transitivité. \\
							Reprenons donc nos dfs restantes : \\
							
						\noindent- (1) idCine $\rightarrow$ adresse, ville \\
						- (2) adresse, ville $\rightarrow$ franchise, nbsalle \\
						- (3) idCine, numSalle $\rightarrow$ SallecompatibleEn3D, nbPlaceStandard, nbPlaceHandicape,nbDbox \\
				 		- (4) idFilm $\rightarrow$ nomFilm, dateSortie \\
						- (5) nomFilm, dateSortie $\rightarrow$ public, idReal, duree, compatible3D \\
						- (6) idFilm, role $\rightarrow$  idAct \\
						- (7) idReal $\rightarrow$ nomR, prenomR \\
						- (8) idAct $\rightarrow$ nomA, prenomA \\
						- (9) idClient $\rightarrow$ nomC, prenomC \\
						- (10) idClient, numReservation $\rightarrow$ nbPlaceStandardRes, nbPlaceHandicapeRes, nbPlaceDBox, idSeance \\
						- (11) idSeance, idCine $\rightarrow$ horaire, dateProjection, numSalle, idFilm, diffusionEn3D \\
						
						A présent, analysons chaque dfs une part une : \\
						
						\noindent - (1) idCine $\rightarrow$ adresse, ville \\
							\\
							\underline{idCine+} \\
							idCIne\\
							\underline{numReservation+} \\
							numReservation \\	
							\\									
						$\rightarrow$ it's OK \\		
							
						\noindent - (2) adresse, ville $\rightarrow$ franchise, nbsalle \\
							\\
							\underline{adresse+} \\
							adresse\\
							\underline{ville+} \\
							ville \\	
							\\									
						$\rightarrow$ it's OK \\
						
						\noindent - (3) idCine, numSalle $\rightarrow$ SallecompatibleEn3D, nbPlaceStandard, nbPlaceHandicape,nbDbox \\
							\\
							\underline{idCine+} \\
							idCIne\\
							adresse\\
							ville\\
							franchise\\
							nbSalle\\
							\underline{numSalle+} \\
							numSalle \\	
							\\									
						$\rightarrow$ it's OK \\													

						\noindent - (4) idFilm $\rightarrow$ nomFilm, dateSortie \\
							\\
							\underline{idFilm+} \\
							idFilm\\	
							\\									
						$\rightarrow$ it's OK \\	
						
						\noindent - (5) nomFilm, dateSortie $\rightarrow$ public, idReal, duree, compatible3D \\
							\\
							\underline{nomFilm+} \\
							nomFilm \\
							\underline{dateSortie+} \\
							dateSortie \\	
							\\									
						$\rightarrow$ it's OK \\	
						
						\noindent - (6) idFilm, role $\rightarrow$  idAct  \\
							\\
							\underline{idFilm+} \\
							idFilm \\
							nomFilm \\
							dateSortie \\
							public \\
							idReal \\
							duree \\
							compatible3D \\
							nomR \\
							prenomR \\
							\underline{role+} \\
							role \\	
							\\									
						$\rightarrow$ it's OK \\																				
			
						\noindent - (7) idReal $\rightarrow$ nomP, prenomP \\
							\\
							\underline{idReal+} \\
							idReal \\
							\\									
						$\rightarrow$ it's OK \\	
						
						\noindent - (8) idAct $\rightarrow$ nomP, prenomP \\
							\\
							\underline{idAct+} \\
							idAct \\	
							\\									
						$\rightarrow$ it's OK \\	

						\noindent - (9) idClient $\rightarrow$ nomC, prenomC \\
							\\
							\underline{idClient+} \\
							idClient \\	
							\\									
						$\rightarrow$ it's OK \\		

						\noindent - (10) idClient, numReservation $\rightarrow$ nbPlaceStandardRes, nbPlaceHandicapeRes, nbPlaceDBox, idSeance \\
							\\
							\underline{idClient+} \\
							idClient \\
							nomC \\
							prenomC \\
							\underline{numReservation+}
							numReservation \\	
							\\									
						$\rightarrow$ it's OK \\		

						\noindent - (11) idSeance, idCine $\rightarrow$ horaire, dateProjection, numSalle, idFilm, diffusionEn3D \\
							\\
							\underline{idSeance+} \\
							idSeance \\
							\\		
							\underline{idCine+}\\
							idCine \\
							adresse \\
							ville \\
							franchise \\
							nbSalle \\							
						$\rightarrow$ it's OK \\							
						\\		
						Ainsi, mettons à jours nos dfs : \\																
						
						\noindent- (1) idCine $\rightarrow$ adresse, ville \\
						- (2) adresse, ville $\rightarrow$ franchise, nbsalle \\
						- (3) idCine, numSalle $\rightarrow$ SallecompatibleEn3D, nbPlaceStandard, nbPlaceHandicape,nbDbox \\
				 		- (4) idFilm $\rightarrow$ nomFilm, dateSortie \\
						- (5) nomFilm, dateSortie $\rightarrow$ public, idReal, duree, compatible3D \\
						- (6) idFilm, role $\rightarrow$  idAct \\
						- (7) idReal $\rightarrow$ nomR, prenomR \\
						- (8) idAct $\rightarrow$ nomA, prenomA \\
						- (9) idClient $\rightarrow$ nomC, prenomC \\
						- (10) idClient, numReservation $\rightarrow$ nbPlaceStandardRes, nbPlaceHandicapeRes, nbPlaceDBox, idSeance \\
						- (11) idSeance, idCine $\rightarrow$ horaire, dateProjection, numSalle, idFilm, diffusionEn3D \\
															
\end{document}